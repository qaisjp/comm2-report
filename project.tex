
% to choose your degree
% please un-comment just one of the following
\documentclass[bsc,frontabs,twoside,singlespacing,parskip,deptreport]{infthesis}     % for BSc, BEng etc.
% \documentclass[minf,frontabs,twoside,singlespacing,parskip,deptreport]{infthesis}  % for MInf

\begin{document}

\title{Build a website to share and discuss software packages}

\author{Qais Patankar}

% to choose your course
% please un-comment just one of the following
% \course{Artificial Intelligence and Computer Science}
%\course{Artificial Intelligence and Software Engineering}
%\course{Artificial Intelligence and Mathematics}
%\course{Artificial Intelligence and Psychology }
%\course{Artificial Intelligence with Psychology }
%\course{Linguistics and Artificial Intelligence}
\course{BEng Computer Science}
%\course{Software Engineering}
%\course{Computer Science and Electronics}
%\course{Electronics and Software Engineering}
%\course{Computer Science and Management Science}
%\course{Computer Science and Mathematics}
%\course{Computer Science and Physics}
%\course{Computer Science and Statistics}

% to choose your report type
% please un-comment just one of the following
%\project{Undergraduate Dissertation} % CS&E, E&SE, AI&L
%\project{Undergraduate Thesis} % AI%Psy
\project{4th Year Project Report}

\date{\today}

\abstract{
These software packages are in the form of ``resources''. Resources are archives containing a metadata files, scripts and other miscellaneous files.

Resources can have assets attached in the form of YouTube videos or screenshots. Users can comment on resources, vote on comments, and it should be possible for authors or other users to respond to comments (similar, but not restricted to, a Q\&A format).

Resources can also be dependent on other resources (dependencies). It should be easy to download a resource (or a tree of resources, when we considered dependencies) — this can either be through the web interface or through a companion CLI.

These resources contain code that may be downloaded and executed by other users, so security must be considered. It must be possible for website administrators to easily (bulk) review resources and leave feedback for resource creators. There could also be some inbuilt heuristics to automatically determine whether or not a resource is dangerous.

Users should be able to showcase the projects they have worked on. It should also be possible for multiple users to collaborate on a project together.
}

\maketitle

\section*{Acknowledgements}
Thank you to my supervisor, Stephen Gilmore, for TODO.

\tableofcontents

%\pagenumbering{arabic}


\chapter{Introduction}

The document structure should include:
\begin{itemize}
\item
The title page  in the format used above.
\item
An optional acknowledgements page.
\item
The table of contents.
\item
The report text divided into chapters as appropriate.
\item
The bibliography.
\end{itemize}

Commands for generating the title page appear in the skeleton file and
are self explanatory.
The file also includes commands to choose your report type (project
report, thesis or dissertation) and degree.
These will be placed in the appropriate place in the title page.

The default behaviour of the documentclass is to produce documents typeset in
12 point.  Regardless of the formatting system you use,
it is recommended that you submit your thesis printed (or copied)
double sided.

The report should be printed single-spaced.
It should be 30 to 60 pages long, and preferably no shorter than 20 pages.
Appendices are in addition to this and you should place detail
here which may be too much or not strictly necessary when reading the relevant section.

\section{Using Sections}

Divide your chapters into sub-parts as appropriate.

\section{Citations}

Note that citations
(like \cite{P1} or \cite{P2})
can be generated using {\tt BibTeX} or by using the
{\tt thebibliography} environment. This makes sure that the
table of contents includes an entry for the bibliography.
Of course you may use any other method as well.

\section{Options}

There are various documentclass options, see the documentation.  Here we are
using an option ({\tt bsc} or {\tt minf}) to choose the degree type, plus:
\begin{itemize}
\item {\tt frontabs} (recommended) to put the abstract on the front page;
\item {\tt twoside} (recommended) to format for two-sided printing, with
  each chapter starting on a right-hand page;
\item {\tt singlespacing} (required) for single-spaced formating; and
\item {\tt parskip} (a matter of taste) which alters the paragraph formatting so that
paragraphs are separated by a vertical space, and there is no
indentation at the start of each paragraph.
\end{itemize}

\chapter{The Real Thing}

Of course
you may want to use several chapters and much more text than here.

\chapter{Selecting HTTP status codes}

\section{401 Unauthorized vs. 403 Forbidden}

We have decided to select the following status codes for the following scenarios:

\begin{itemize}
  \item 401: being unauthenticated for a request that requires authentication
  \item 403: being authenticated but not authorized to perform an action
\end{itemize}

This might seem obvious, but the status text for \emph{401} is \emph{401 Unauthorized},
despite it being for authentication and not authorization.

Sources:
\begin{itemize}
  \item https://stackoverflow.com/questions/3297048/403-forbidden-vs-401-unauthorized-http-responses
  \item https://httpstatuses.com/401
  \item https://httpstatuses.com/403
\end{itemize}


% use the following and \cite{} as above if you use BibTeX
% otherwise generate bibtem entries
\bibliographystyle{plain}
\bibliography{mybibfile}

\end{document}
